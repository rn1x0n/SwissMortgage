\nonstopmode{}
\documentclass[a4paper]{book}
\usepackage[times,inconsolata,hyper]{Rd}
\usepackage{makeidx}
\usepackage[utf8,latin1]{inputenc}
% \usepackage{graphicx} % @USE GRAPHICX@
\makeindex{}
\begin{document}
\chapter*{}
\begin{center}
{\textbf{\huge Package `SwissMortgage'}}
\par\bigskip{\large \today}
\end{center}
\begin{description}
\raggedright{}
\item[Title]\AsIs{Tools for mortgage calculations in Switzerland}
\item[Description]\AsIs{}
\item[Version]\AsIs{0.1}
\item[Author]\AsIs{Richard Nixon }\email{richard2718@gmail.com}\AsIs{}
\item[Maintainer]\AsIs{Richard Nixon }\email{richard2718@gmail.com}\AsIs{}
\item[Depends]\AsIs{R (>= 2.15.1), ggplot2, plyr}
\item[License]\AsIs{GPL-2}
\item[Collate]\AsIs{'interest.R'}
\end{description}
\Rdcontents{\R{} topics documented:}
\inputencoding{utf8}
\HeaderA{amortization}{Find the value of one amortization payment}{amortization}
%
\begin{Description}\relax
This function finds the value of one amortization
payment, given the current total debt, the current annual
interest rate and the period over which to pay the debt.
\end{Description}
%
\begin{Usage}
\begin{verbatim}
  amortization(debt = 100, rate = 1, period = 5)
\end{verbatim}
\end{Usage}
%
\begin{Arguments}
\begin{ldescription}
\item[\code{debt}] Current debt

\item[\code{rate}] Annual percentage rate

\item[\code{period}] Period of amortization repayments in years
\end{ldescription}
\end{Arguments}
%
\begin{Value}
The amortizaiton for one payment
\end{Value}
%
\begin{Examples}
\begin{ExampleCode}
amortization(debt = 1000, rate = 1, period = 5)
\end{ExampleCode}
\end{Examples}
\inputencoding{utf8}
\HeaderA{fix.rate}{Find the fixed interest rates over given future period}{fix.rate}
%
\begin{Description}\relax
This function predicts the fixed interest rates for a
future time, given the fixed rates now, and the flexibles
rates in the future
\end{Description}
%
\begin{Usage}
\begin{verbatim}
  fix.rate(start.time = 0, period = 5,
    current.fix.rates = NULL, flex.rate = NULL)
\end{verbatim}
\end{Usage}
%
\begin{Arguments}
\begin{ldescription}
\item[\code{start.time}] the time in years when the fixed rate
mortgage will start

\item[\code{period}] the fixed period for the mortgage

\item[\code{current.fix.rates}] vector of length n giving the
current fixed rate mortges for a fixed period of 1 to n
years

\item[\code{flex.rate}] object from \code{\LinkA{flex.rate}{flex.rate}}
giving the flexible rates. The period of this must be at
least as long as \code{start.time}
\end{ldescription}
\end{Arguments}
%
\begin{Value}
The fixed interest rate
\end{Value}
%
\begin{Examples}
\begin{ExampleCode}
flexRate <- flex.rate()
current.fix.rates <- c(0.980, 0.960, 1.020, 1.150, 1.300, 1.460, 1.620, 1.780, 1.920, 2.060)
fix.rate(start.time = 0, period = 5, current.fix.rates = current.fix.rates, flex.rate = flexRate)
\end{ExampleCode}
\end{Examples}
\inputencoding{utf8}
\HeaderA{flex.rate}{Find the flexible interest rates over given future period}{flex.rate}
%
\begin{Description}\relax
This function predicts the flexible interest rates for a
given period.
\end{Description}
%
\begin{Usage}
\begin{verbatim}
  flex.rate(current.rate = 1, final.rate = 5, period = 10,
    extend.period = 50)
\end{verbatim}
\end{Usage}
%
\begin{Arguments}
\begin{ldescription}
\item[\code{current.rate}] current interest rate

\item[\code{final.rate}] final interest rate

\item[\code{period}] period in years over which the rate will
change

\item[\code{extend.period}] after the period has ended keep at
the final rate for this long
\end{ldescription}
\end{Arguments}
%
\begin{Value}
A data frame with elements \begin{ldescription}
\item[\code{month}] time in months
\item[\code{rate}] interest rate that month
\end{ldescription}
\end{Value}
%
\begin{Examples}
\begin{ExampleCode}
flex.rate(current.rate = 1, final.rate = 5, period = 10)
\end{ExampleCode}
\end{Examples}
\inputencoding{utf8}
\HeaderA{interest.pay}{Find the payments over time for a mortgage}{interest.pay}
%
\begin{Description}\relax
This function finds the payment each month for a fixed or
amortization mortgage. This is also broken down by how
much is a repayment and how much is interest.
\end{Description}
%
\begin{Usage}
\begin{verbatim}
  interest.pay(debt = 100, rate = 1, period = 1,
    interest.only = TRUE, amortization.period = 20)
\end{verbatim}
\end{Usage}
%
\begin{Arguments}
\begin{ldescription}
\item[\code{debt}] the amount borrowed

\item[\code{rate}] either a single value giving the annual
percentage rate, or a vector of length \code{12*period}
giving the interest rate by month

\item[\code{period}] period (years) to calculate mortgage for

\item[\code{interest.only}] logical. If TRUE perform
calculaitons for an interest only mortgage, if FALSE then
amoritize

\item[\code{amortization.period}] the repayment period of an
amortization mortgage
\end{ldescription}
\end{Arguments}
%
\begin{Value}
A data frame with elements \begin{ldescription}
\item[\code{month}] time in months
\item[\code{interest}] interest that month
\item[\code{repayment}] repayment that month\item[\code{payment}] sum
of interest and repayment
\end{ldescription}
\end{Value}
%
\begin{Examples}
\begin{ExampleCode}
interest.pay(debt = 1000, rate = 1, period = 5, interest.only = FALSE, amortization.period = 20)
\end{ExampleCode}
\end{Examples}
\inputencoding{utf8}
\HeaderA{plan.pay}{Find the payments over time for a set of mortgages}{plan.pay}
%
\begin{Description}\relax
This function finds the payment each month for a set of
fixed or amortization mortgages. This is broken down by
how much is a repayment and how much is interest, for
each mortgage
\end{Description}
%
\begin{Usage}
\begin{verbatim}
  plan.pay(plan)
\end{verbatim}
\end{Usage}
%
\begin{Arguments}
\begin{ldescription}
\item[\code{plan}] list giving details of a set of mortgages,
see Details
\end{ldescription}
\end{Arguments}
%
\begin{Details}\relax
The set of mortages is defined by the list \code{plan}.
Each element of ths list is a named list of lists
defining a set of subsequent mortgages. The bottom level
list has elements needed for \code{\LinkA{interest.pay}{interest.pay}}
\end{Details}
%
\begin{Value}
A long format data frame with one row per mortgage per
month with elements \begin{ldescription}
\item[\code{month}] time in months
\item[\code{mortgage}] factor giving the name of the mortgage,
taken from the names in \code{plan}
\item[\code{interest}] interest that month and mortgage
\item[\code{repayment}] repayment that month and mortgage
\item[\code{payment}] sum of interest and repayment for that
month and mortgage
\end{ldescription}
\end{Value}
%
\begin{Examples}
\begin{ExampleCode}
plan <- list(
 "Fix1" = list(
   list(debt = 1000, rate = 1, period = 5, interest.only = TRUE, amortization.period = NULL),
   list(debt = 1000, rate = 2, period = 3, interest.only = TRUE, amortization.period = NULL)
 ),
 "Amortization" = list(
   list(debt = 1000, rate = 2, period = 8, interest.only = FALSE, amortization.period = 20)
 )
)

plan.pay(plan)
\end{ExampleCode}
\end{Examples}
\inputencoding{utf8}
\HeaderA{ribbon.plot.pay}{Ribbon plot for payments}{ribbon.plot.pay}
%
\begin{Description}\relax
This function
\end{Description}
%
\begin{Usage}
\begin{verbatim}
  ribbon.plot.pay(pay, y = "payment", xmax = NULL,
    ymax = NULL)
\end{verbatim}
\end{Usage}
%
\begin{Arguments}
\begin{ldescription}
\item[\code{pay}] an object from plan.pay

\item[\code{y}] y value to plot, either "payment", "interest",
"repayment"

\item[\code{xmax}] optional x-axis upper limit

\item[\code{ymax}] optional y-axis upper limit
\end{ldescription}
\end{Arguments}
%
\begin{Value}
A ggplot2 ribbon plot
\end{Value}
%
\begin{Examples}
\begin{ExampleCode}
plan <- list(
 "Fix1" = list(
   list(debt = 1000, rate = 1, period = 5, interest.only = TRUE, amortization.period = NULL),
   list(debt = 2000, rate = 2, period = 3, interest.only = TRUE, amortization.period = NULL)
 ),
 "Amortization" = list(
   list(debt = 1000, rate = 2, period = 8, interest.only = FALSE, amortization.period = 20)
 )
)

plan <- plan.pay(plan)
ribbon.plot.pay(plan)
\end{ExampleCode}
\end{Examples}
\inputencoding{utf8}
\HeaderA{shinyPlan2plan}{Convert a shinyPlan list to a plan list}{shinyPlan2plan}
%
\begin{Description}\relax
This function converts a shinyPlan list to a more general
plan list which can be processed by
\code{\LinkA{plan.pay}{plan.pay}}
\end{Description}
%
\begin{Usage}
\begin{verbatim}
  shinyPlan2plan(shinyPlan, currentFixRates, flexRate,
    timeHorizon = NULL)
\end{verbatim}
\end{Usage}
%
\begin{Arguments}
\begin{ldescription}
\item[\code{shinyPlan}] a shiny plan list

\item[\code{currentFixRates}] vector of length n giving the
current fixed rate mortges for a fixed period of 1 to n
years

\item[\code{flexRate}] object from \code{\LinkA{flex.rate}{flex.rate}}
giving the flexible rates.

\item[\code{timeHorizon}] period over which the final set of
morgages perdict over. Default is NULL, in which case it
is the longest single period of the shinyPlan set
\end{ldescription}
\end{Arguments}
%
\begin{Details}\relax
shinyPlan is a list of named lists with elements
\begin{itemize}
 \item debtThe amount borrowed
\item fix.rateTRUE for a fixed rate of interest, FALSE
for an interest rate that can change over time. These are
calculated from \code{current.fix.rate} and
\code{flex.rate} \item periodPeriod of mortgage in
years \item interest.onlyIf TRUE then only interest is
paid on the debt, if FALSE then the debt is amortized
\item amortization.periodif \code{interest.only = TRUE}
then this gives the amortization period in years
\item renewThe period to keep renewing the mortgage
for. 0 means the mortgage is not renewed 
\end{itemize}

\end{Details}
%
\begin{Value}
a plan list. See \code{\LinkA{plan.pay}{plan.pay}}
\end{Value}
%
\begin{Examples}
\begin{ExampleCode}
shinyPlan <- list(
   "Fix1" = list(debt = 200000, fix.rate = TRUE, period = 3, interest.only = TRUE, amortization.period = NULL, renew = 5),
   "Fix2" = list(debt = 200000, fix.rate = TRUE, period = 3, interest.only = TRUE, amortization.period = NULL, renew = 0),
   "Amm1" = list(debt = 200000, fix.rate = FALSE, period = 10, interest.only = FALSE, amortization.period = 20, renew = 0)
)
currentFixRates <- c(0.980, 0.960, 1.020, 1.150, 1.300, 1.460, 1.620, 1.780, 1.920, 2.060)
flexRate <- flex.rate()
shinyPlan2plan(shinyPlan = shinyPlan, currentFixRates = currentFixRates,  flexRate = flexRate, timeHorizon = 10)
\end{ExampleCode}
\end{Examples}
\printindex{}
\end{document}
